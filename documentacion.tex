\input{preambuloSimple.tex}

\title{	
	\normalfont \normalsize 
	\textsc{\textbf{Inteligencia de Negocio (2017-2018)} \\ Grado en Ingeniería Informática \\ Universidad de Granada} \\ [25pt] 
	\horrule{0.5pt} \\[0.4cm]
	\huge Práctica 2 \\
	\horrule{2pt} \\[0.5cm]
}

\author{Juan José Sierra González \\ jjsierra103@gmail.com}

\date{\normalsize\today}

\begin{document}
	\maketitle
	\thispagestyle{empty}
	
	\newpage
	
	\tableofcontents
	
	\listoffigures
	
	\listoftables
	
	\newpage
	
	\section{Introducción}
	En esta práctica se tratará de interpretar el comportamiento de distintos algoritmos de clustering, trabajando sobre una base de datos de accidentes en España en el año 2013. Eligiendo distintos subconjuntos del conjunto total de datos se elaborarán 3 casos de estudio sobre los que ejecutar los algoritmos. Estos algoritmos de clustering seleccionados para el estudio son \textbf{K-Means, DBSCAN, Spectral Clustering, Birch y Ward}.
	
\end{document}