%%%%%%%%%%%%%%%%%%%%%%%%%%%%%%%%%%%%%%%%%
% Short Sectioned Assignment LaTeX Template Version 1.0 (5/5/12)
% This template has been downloaded from: http://www.LaTeXTemplates.com
% Original author:  Frits Wenneker (http://www.howtotex.com)
% License: CC BY-NC-SA 3.0 (http://creativecommons.org/licenses/by-nc-sa/3.0/)
%%%%%%%%%%%%%%%%%%%%%%%%%%%%%%%%%%%%%%%%%

%----------------------------------------------------------------------------------------
%	PACKAGES AND OTHER DOCUMENT CONFIGURATIONS
%----------------------------------------------------------------------------------------

\documentclass[paper=a4, fontsize=11pt]{scrartcl} % A4 paper and 11pt font size

% ---- Entrada y salida de texto -----

\usepackage{array}
\usepackage{booktabs}
\usepackage{tabulary}
\usepackage{multirow}

\usepackage{color}
\usepackage{listings}
\usepackage[T1]{fontenc} % Use 8-bit encoding that has 256 glyphs
\usepackage[utf8]{inputenc}
%\usepackage{fourier} % Use the Adobe Utopia font for the document - comment this line to return to the LaTeX default
\usepackage{footnote}
\usepackage{eurosym}
\usepackage{eurosym} % Sirve para poner \euro y que salga el símbolo del €.

% ---- Idioma --------
 
\usepackage[spanish, es-tabla]{babel} % Selecciona el español para palabras introducidas automáticamente, p.ej. "septiembre" en la fecha y especifica que se use la palabra Tabla en vez de Cuadro

% ---- Otros paquetes ----
\usepackage{subfig}
\usepackage{url} % ,href} %para incluir URLs e hipervínculos dentro del texto (aunque hay que instalar href)
\usepackage{amsmath,amsfonts,amsthm} % Math packages
%\usepackage{graphics,graphicx, floatrow} %para incluir imágenes y notas en las imágenes
\usepackage{graphics,graphicx, float} %para incluir imágenes y colocarlas

% Para hacer tablas comlejas
%\usepackage{multirow}
%\usepackage{threeparttable}

%\usepackage{sectsty} % Allows customizing section commands
%\allsectionsfont{\centering \normalfont\scshape} % Make all sections centered, the default font and small caps

\usepackage{fancyhdr} % Custom headers and footers
\pagestyle{fancyplain} % Makes all pages in the document conform to the custom headers and footers
\fancyhead{} % No page header - if you want one, create it in the same way as the footers below
\fancyfoot[L]{} % Empty left footer
\fancyfoot[C]{} % Empty center footer
\fancyfoot[R]{\thepage} % Page numbering for right footer
\renewcommand{\headrulewidth}{0pt} % Remove header underlines
\renewcommand{\footrulewidth}{0pt} % Remove footer underlines
\setlength{\headheight}{13.6pt} % Customize the height of the header

\numberwithin{equation}{section} % Number equations within sections (i.e. 1.1, 1.2, 2.1, 2.2 instead of 1, 2, 3, 4)
\numberwithin{figure}{section} % Number figures within sections (i.e. 1.1, 1.2, 2.1, 2.2 instead of 1, 2, 3, 4)
\numberwithin{table}{section} % Number tables within sections (i.e. 1.1, 1.2, 2.1, 2.2 instead of 1, 2, 3, 4)

\setlength\parindent{0pt} % Removes all indentation from paragraphs - comment this line for an assignment with lots of text

\newcommand{\horrule}[1]{\rule{\linewidth}{#1}} % Create horizontal rule command with 1 argument of height


\title{	
	\normalfont \normalsize 
	\textsc{\textbf{Inteligencia de Negocio (2017-2018)} \\ Grado en Ingeniería Informática \\ Universidad de Granada} \\ [25pt] 
	\horrule{0.5pt} \\[0.4cm]
	\huge Práctica 2 \\
	\horrule{2pt} \\[0.5cm]
}

\author{Juan José Sierra González \\ jjsierra103@gmail.com}

\date{\normalsize\today}

\begin{document}
	\maketitle
	\thispagestyle{empty}
	
	\newpage
	
	\tableofcontents
	
	\listoffigures
	
	\listoftables
	
	\newpage
	
	\section{Introducción}
	En esta práctica se tratará de interpretar el comportamiento de distintos algoritmos de clustering, trabajando sobre una base de datos de accidentes en España en el año 2013. Eligiendo distintos subconjuntos del conjunto total de datos se elaborarán 3 casos de estudio sobre los que ejecutar los algoritmos. Estos algoritmos de clustering seleccionados para el estudio son \textbf{K-Means, DBSCAN, Spectral Clustering, Birch y Ward}.
	
	\section{Caso de estudio 1}
	
	\subsection{¿Qué se analiza?}
	En el primero de estos casos de estudio se van a analizar los accidentes que han tenido lugar en autovías y autopistas. Se comprobará cuántos vehículos suelen verse involucrados en estos accidentes y cómo se distribuyen las cifras de heridos y víctimas mortales.
	
	\subsection{Resultados de los algoritmos}
	A continuación se muestra una tabla que refleje los resultados de los distintos algoritmos para este primer caso de estudio. En la tabla se incluyen el número de clusters en los que el algoritmo ha segmentado los ejemplos, el tiempo de ejecución y los scores obtenidos según las métricas de Calinski-Harabaz y Silhouette.
	
	\begin{table}[H]
		\centering
		\resizebox{\textwidth}{!}{
			$\begin{tabular}{ *{12}{c} }
			\toprule
			\textbf{Algorithm} && \textbf{Clusters} && \textbf{Execution time} && \textbf{CH Score} && \textbf{Silhouette Score}\\
			\midrule
			K-Means && 4 && 0.682 && 7362.737 && 0.76754\\
			DBSCAN && 16 && 0.209 && 221.446 && 0.14467\\
			Birch && 4 && 0.151 && 4425.728 && 0.70268\\
			Spectral Clustering && 4 && 4.523 && 5655.098 && 0.72585\\
			Ward && 100 && 0.550 && 0.000 && 0.00000\\
			\bottomrule
			\end{tabular}$
		}
		\caption{Resultados de los algoritmos de clustering para el primer caso de estudio.}
	\end{table}
	
\end{document}